\documentclass[a4paper]{article}
\usepackage[a4paper,includeheadfoot,margin=1.5cm]{geometry}
\usepackage{shorttoc,hyperref,fancyhdr,lastpage}
\pagestyle{fancy}
\renewcommand{\headrulewidth}{0.5pt}
\fancyhead[L]{}
\fancyhead[C]{\textbf{{\large Electronicity - General Rules}}} 
\fancyhead[R]{}
\fancyfoot[L]{}
\fancyfoot[C]{}
\fancyfoot[R]{\textsl{Page \thepage{} in \pageref{LastPage}}}

\hypersetup{colorlinks=true, urlcolor=blue, linkcolor=black}

\begin{document}
\begin{titlepage}
	\begin{center}
		\vspace*{5cm}
		\textsl{{\Huge \textbf{Electronicity}\\\vspace*{.5cm}General Rules}}\\
		\vspace*{1cm}
		\rule{.7\linewidth}{0.5mm}\\
		\vspace*{1cm}
		\textsl{{\large \today}}\\
	\end{center}
\end{titlepage}
\tableofcontents
\pagebreak
\section{Introduction}
Welcome on \textbf{Electronicity}!\\
This group, hosted on the Discord platform, is an online semi-private place to talk about electronics in general, programming, 3D printing ... and share knowledge and project ideas.\\
\\
It aims at gathering people interested in new technologies and \textsl{do it yourself {\footnotesize (diy)}} making: most of the members here are hobbyists or professionals, willing to help beginners and to share their projects.\\
\\
This community is permanently growing and you can help its development by inviting others with the following link: \url{https://discord.me/electronicity}\\
\\
The rules defined in this document have to be respected by each of the members in this group. If one member does not respect them, he/she is exposed to sanctions, as stated in \hyperref[sec:sanctions]{\textbf{Sanctions and appeal}}.\\\\
These rules are not meant to be exhaustive, and moderators and administrators reserve the right to determine what conduct is considered to be forbidden on Electronicity and to take action against this kind of activity.\\
If you find yourself in an ambiguous situation that is not covered by any rule in this document, ask a staff member to get a clear answer and a new rule will be added.\\
\\
These set of rules is internal to the Electronicity server and does \textbf{not} replace the Discord Terms of Service.\\
\textbf{As a Discord user, you must respect the Discord Terms of Service (\url{https://discordapp.com/terms}) and you are responsible for your actions on the platform, in any server you are.}\\
\\
\textbf{You are not allowed to ignore the rules set out in this document and you must respect them.}\\
If any change occurs, the members will be notified through the use of an \texttt{@everyone} mention in an appropriate public channel, however not receiving this notification will not be an acceptable excuse.
\pagebreak

\section{Help}

\subsection{How to ask for help}
When you are seeking some help with one project, in any subject, you are allowed to ask other members in public chatrooms or via private messaging. However you should always respect the following points:\\\\
$\bullet$ Other members do not have the duty to help you: they are volunteers willing to help. You should therefore be polite and respect their work: \textsl{Please} and \textsl{Thank you} are very popular expressions which should be included in any help request you make.\\\\
$\bullet$ It is unnecessary to ask if you can ask: directly expose your problem politely and other members are likely to answer within minutes.\\\\
$\bullet$ Keep in mind that if questions are asked in public channels, experts will be more likely to give an answer quickly and the answer could also be useful to some other members with similar issues.\\\\
$\bullet$ You may mention one "Expert" role in order to get help faster from people who can potentially help you.\\\\
$\bullet$ Help cannot be asked anywhere on the server. You should not ask for help in the following channels :\\
$\cdot$ any channel in the "\texttt{INFOS}" category\\
$\cdot$ any channel in the "\texttt{SERVER}" category\\
$\cdot$ \texttt{\#projects}\\
$\cdot$ \texttt{\#suggestions}\\
$\cdot$ \texttt{\#bot\_ideas}\\
$\cdot$ \texttt{\#voice\_chat}\\\\
$\bullet$ If possible, avoid asking help in \texttt{\#general} and rather ask in \texttt{\#help}: this channel is the default one dedicated to help, whichever the topic is.\\\\
$\bullet$ Do not ask for help in the middle of another active discussion: wait for this discussion to end or ask in a different channel if your problem includes multiple topics.

\subsection{How to help}
When you have the solution to a problem exposed by another user, feel free to help him/her.\\
Some points should be respected though:\\\\
$\bullet$ You should not give direct answers with no explaination at all: you should always explicitely give the reason why a result should be this way instead of another. Progress is made when people understand how it works, not when they make it work with pre-made solutions.\\\\
$\bullet$ When you feel a user is seeking help for his/her homework, you should not give answers but explanations. Homework is made to practice and students should always do their exercises by themselves.\\
For instance, give the mathematical law which should be applied instead of the numerical result which should be found.\\ \textsl{This point can be ignored if a user brings his/her own results and asks for a verification.}\\\\
$\bullet$ Keep in mind that everything on the internet is not necessarily true. Some websites are more serious than others and you should always check that the information given on such websites is reliable before sending their link in the discussion.\\\\
$\bullet$ Everyone has to be a beginner before mastering a subject: "\textsl{your code is shit}", "\textsl{your circuit is the worst I've ever seen}" and other discouraging remarks should never appear in a conversation.\\
When you have ideas of improvements for a code/project, explain them without insulting others for their relatively poor knowledge.
\pagebreak

\section{Interaction with other users}
It is not always easy to interact peacefully with other humans on the Internet, especially when different ideas are shared. In order to make the chat stay calm and comfortable, the following rules must be respected:

\subsection{Harassment}
It is illegal to harass other users. Harassment is forbidden on Discord, as stated in the Discord ToS and will be reported if found on the Electronicity server.

\subsection{Defamation}
Defamation, which consists in the the communication of a false statement that harms the reputation of a user or group of users, is not allowed on Discord, nor Electronicity.\\
Defamation is forbidden by the Discord ToS and will be reported if found on the Electronicity server.

\subsection{Discrimination}
\textbf{\textsl{All Discord users are created equal.}}\\
Discrimination based on pseudonyms, age, sex, gender, nationality or any characteristic that may differ between users will not be tolerated on Electronicity.

\subsection{Threats, insults and intimidation}
As for harassment and defamation, threats, insults and inimidation are not allowed on Electronicity.\\
However, since sending \textsl{fake} threats or insults can be used as a form of second degree humour, this practice will only be tolerated when it is clear that the concerned users are close friends or when this happens in the middle of a conversation where anyone can understand it is not serious.

\subsection{Private information}
Discord makes it really easy to hide behind a pseudonym: users who want to stay anonymous do not need to share their private information publicly.\\
It is therefore forbidden to attempt to obtain such information about a user through any malicious manner.\\
If a user decides to share some information about him/her via private messages, you are not allowed to send this information publicly without his/her explicitly given permission.\\
\\
Some information are too sensitive and should never be neither asked for, given nor released in public channels, such as passwords or user/bot tokens.

\subsection{Userbots and selfbots}
Userbots and selfbots are forbidden on our server as well as on the whole Discord platform, as they are likely to abuse the Discord API.\\
If one userbot is found on Electronicity, it will immediately be banned and reported to Discord.\\
If one selfbot is found on Electronicity, the user will be asked to disable it, then if he/she refuses, he is exposed to sanctions.\\
\\
\textsl{{\small \textbf{Notes:}\\
$\cdot$ A \textbf{userbot} refers to a discord user account (without the \texttt{BOT} label next to its name) which is automatically managed by an application or software (not human at all).\\
$\cdot$ A \textbf{selfbot} refers to a normal human user who uses an application or software to interact with the Discord API directly from his/her Discord account.}}
\pagebreak

\section{Text chat}
Discord provides the ability to talk with others via text messages. In order to facilitate the use of the text chatrooms, you should always respect the following points:

\subsection{Text formatting}
Feel free to use text formatting such as italic, bold or underlined text to emphasize the content of your messages.\\
You may also use code blocks with syntax highlighting to bring colors to the chat!\\
Any piece of code should either be shared via a code-sharing platform such as pastebin or within a code block.\\
Text-to-Speech (TTS) messages have been disabled on the server as they appear to be really annoying when different languages are mixed.

\subsection{Language}
The only language allowed on this server is the English language, except in the channels of the "\texttt{INTERNATIONAL}" category, where the allowed languages are specified in each channel's topic.

\subsection{Attacks}
The following actions are considered to be attacks against Electronicity and are thus forbidden:\\\\
$\bullet$ \textbf{Spam}: Spamming is defined as sending messages repeatedly and very quickly. Sending multiple short messages will be considered as spamming at some point, even if they are part of a correct conversation. Please use \texttt{Shift+Enter} to force a line break and gather your small sentences in a same unique message.\\\\
$\bullet$ \textbf{Raids}: Raiding a server is the action of joining it in order to spam and/or annoy its users. Raids are often conducted by multiple users simultaneously and are forbidden by the Discord ToS.\\\\
$\bullet$ \textbf{Fully capitalized sentences}: repeatedly using fully capitalized words or sentences is a poor way to attract human's eye and attention. As a part of unpopular and widely used clickbait techniques, fully capitalized content will be removed.

\subsection{Pseudonyms and avatars}
You are free to set your own username and avatar like you want. However if you want to interact with other users correctly, you should follow these recommendations:\\\\
$\bullet$ Make sure you are mentionnable by other users. You can achieve this by including a few ASCII characters in your nickname: no need to change you Discord username, you can set a different and specific nickname on Electronicity.\\\\
$\bullet$ Make sure your avatar does not contain any illegal graphics such as (pedo)pornographical or gore elements: this is illegal and you risk being reported to Discord if you do not change it to a less edgy one.

\subsection{Bots' usage}
Multiple robots can be used on the server to improve your Discord experience. Their use has some limitations though:\\\\
$\bullet$ Bot commands must only be used in the channel named "\texttt{\#bot\_commands}". Using the same bot command will not be considered as a form of spam as long as it happens in this specific channel.\\\\
$\bullet$ Electronicity has two official robots: Ariana and Electroid. They have some really useful features such as a resistance/color code translator. Yet, since they are hosted on a Raspberry Pi, they might be offline sometimes or crash when they have too much data to process. Any abuse of their commands will thus be sanctionned more easily than other bots.\\\\
\subsection{Conversation subjects}
Electronicity is a Discord group to discuss about any topic related to electronics, CNC machining (3D printing, laser cutting, milling ...), coding and any other similar topic.\\\\
The following points should always be respected:\\\\
$\bullet$ Respect the channel topics. They are either obvious as channels are named after them, or listed in channels' descriptions.\\
If you are unsure about where to talk about a specific subject, ask a staff member.\\
\texttt{\#general} has no particular subject: any conversation can take place there, as long as it respects the other points of this section.\\\\
$\bullet$ \textsl{Internet memes} are tolerated in conversations if they do not break of flood them.\\
When you want to share funny or \textsl{non-electronics-related} content, please use \texttt{\#offtopic}.\\\\
$\bullet$ Do not ask or explain how to create illegal machines or devices.\\ Talking about the physics behind their working principle is accepted, but describing how to create one is not.\\\\
\textsl{{\small \textbf{Note:}\\
RF jammers, powerful electric or electromagnetic guns (tasers, coilguns), extremely high power lasers, flamethrowers or any other type of weapon or harmful device have restricted uses is most countries and are thus considered to be "illegal devices".}}

\pagebreak

\section{Voice chat}
Electronicity has a few voice channels gathered in a category named "\texttt{VOICE CHANNELS}". These channels can be used at any time and do not have any maximum users limitation (except the one named "\texttt{1 to 1}").

\subsection{Resources}
As it is dificult - if not impossible! - to share links, images, documents or pieces of code in a vocal conversation, you can throw all these various resources in the text channel named "\texttt{\#voice\_chat}".

\subsection{Behaviour}
As vocal channels are not supervised, the moderators and administrators hope you will be intelligent enough to respect others and stay polite to them, by adjusting the threshold of your microphone and not yelling, for instance.\\\\
If you witness somebody behaving agressively or impolitely in voice channels, please report him/her to a moderator or administrator. Yet, since it is impossible to obtain a proof of what happened, a sanction will only be applied if the user is reported multiple times or by different users.
\pagebreak

\section{External content}
You may sometimes need to send files or images during a conversation. This is not forbidden as long as the files do not:\\\\
$\bullet$ contain any virus, Trojan horse, worm, time bomb, cancelbot, corrupted data, or any other similar software or program that may represent a threat for other users' computers\\\\
$\bullet$ hold any obscene, gore, pornographic, sexual or any other illegal information or content\\\\
$\bullet$ violate the intellectual property protected by a copyright, trademark or patent\\\\
If one of the files may be helpful to multiple users, feel free to post it in \texttt{\#resources}, with a mention of the appropriate "Beginner" role
\pagebreak

\section{Links and advertisement}

\subsection{URLs}
You are allowed to post URLs and hyperlinks in the different text discussions as long as:\\\\
$\bullet$ they do not redirect users to a malicious, phishing, illegal, pornographical or controversial website\\\\
$\bullet$ the linked websites contain reliable information\\\\
$\bullet$ they are not off-topic compared to the current discussion of the channel.\\\\
In order to protect the discussions from being flooded by excessively and unnecessarily long URLs, a robot will automatically shorten any link that exceeds 100 characters, thanks to the bit.ly API.\\
Avoiding the shortening system too often will result in the deletion of your messages containing huge URLs.\\\\
As for files, if you estimate that one link may be helpful to multiple users, feel free to post it in \texttt{\#resources}, with a mention of the appropriate "Beginner" role.

\subsection{Advertisement}
It is forbidden to advertise for other Discord groups in public channels. You can ask an administrator if you want to partner with Electronicity and get your link permanently displayed in a public channel as a resource.\\\\
Advertisement for your YouTube, Steam, Twitch ... channel is forbidden as well, unless it is related to the discussion the link is posted in.
\pagebreak

\section{Roles and mentions}

\subsection{Roles}

\subsubsection{The member role}
When entering the server, you will need to go through a verification procedure to make sure you are not a userbot: type \textsl{I agree} in the channel named "\texttt{WELCOME}" and you will automatically be given the \textsl{@Member} role by a robot.

\subsubsection{Self-assignable roles}
Multiple roles are self-assignable and you can get them by using the correct command.\\\\
\textbf{Expert roles} do not only concern professionals: anyone who is willing to help can request them to get a notification when someone needs help.\\\\
\textbf{Beginner roles} are for people who want to get notified when a new resource is posted or who want to show they do not master a certain topic yet but try to improve. Having one or multiple Beginner role(s) may affect the way people give you explanations.\\\\
All the roles that are self-assigable are also self-removable.\\
Since this action is performed by the robot, feel free to ask a moderator or administrator when it is offline and does not respond to commands.

\subsubsection{Non-self-assignable roles}
Some other roles are not self-assignable and can only be obtained from a staff member:\\\\
\textbf{Electronicitizen} is reserved to active members\\\\
\textbf{Moderator} gives huge extra permissions such as the ability to kick and ban users. It will therefore be only given to trusted members who already own the Electronicitizen role.\\\\
\textbf{Administrator} is the highest role on this server and can only be given by the server owner. There are already two administrators on this server, which is enough at the moment.\\\\
Do not ask for any role that is not self-assignable.

\subsection{Mentions}
Some roles are not mentionnable in order to prevent any "notification-spam". Making the \texttt{@Member} role mentionnable would be like replacing the disabled \texttt{@everyone} mention, which is not the original point.\\\\
You may mention some users or roles when you want their help or attention, but you shouldn't abuse of this functionnality.\\
Staff members, as well as members with the Electronicitizen role, are allowed to use both \texttt{@everyone} and \texttt{@here} commands. If you do not want to get a notification when these commands are used, you can disable them in your notification settings for Electronicity.
\pagebreak

\section{Staff members' status}
The service provided by Discord lets servers have one unique and omnipotent owner, giving them the structure of a dictature rather than a republic. You must therefore listen to moderators and administrators as they always have the upper hand and can terminate any argument with different moderation tools.\\
\\
Moderators and administrators are volunteers who accept to spend a bit of their precious spare time to manage the community and make sure the chat stays comfortable at any time. They are normal users: feel free to have conversations with them like with any other member. However, when a problem is spotted or reported, they will take decisions quickly to remove it.\\
They are here to prevent uncomfortable and annoying situations from happening and you must respect their work.\\
\\
Since issues need to be solved as fast as possible to prevent them from growing and involving more users, some decisions may look unfair at first sight.\\
Do not criticize the decisions that are made by moderators and administrators: this is not your job to tell what you consider to be fair or not. The only exception to this rule is when you are involved and want to appeal a decision.
\pagebreak
\section{Sanctions, reports, complaints and appeals}
\label{sec:sanctions}

\subsection{Sanctionning procedure}
When the relations between multiple users become violent and lead to a massive negative activity on Electronicity, a specific procedure will be followed until the final verdict is given.\\
\\
Electronicity aims at being as fair as possible, but in some situations a solution has to be found quickly and sanctions may be applied directly by moderators and administrators, without respecting the procedure described below.  This has to be applied in emergency situations only and an investigation will be conducted by the owner every time it happens to prevent any abuse from the moderators.\\
\\
The normal procedure that should always be respected is the following:\\\\
$\bullet$ Any user involved in the case is insulated from the rest of the server, in a quarantine chatroom (both guilty and innocent users). Being in quarantine is only a temporary status and should not be considered as a sanction.\\\\
$\bullet$ All the users in quarantine can debate peacefully and expose their point of view, bringing proofs such as screenshots.\\\\
$\bullet$ The staff members gather in the staff channel to try to find a fair solution to the situation.\\\\
$\bullet$ The staff members expose their common solution to the users in quarantine. The debate continues until a final and fair solution is found.\\\\
$\bullet$ The users who are found innocent are released from quarantine and the debate continues shortly with the guilty one(s) in the quarantine channel (especially to understand why they acted this way). They are informed of the sanction that is going to be applied, then when they are calm, they are released from quarantine.\\\\
$\bullet$ If some new elements are found after a short investigation or if a user is unhappy with the final verdict, an appeal procedure can be started, unless one administrator or moderator immediately refuses it, bringing a valid justification.

\subsection{Sanctions}
When a user is found guilty of violating any of the rules set out in this document, he is exposed to the following sanctions:

\subsubsection{Warning messages}
Warning messages are the softest sanctions of this list. They are sent by moderators to remind you that this community has some rules which should be respected.\\
Even if they only consist in simple messages, do not ignore them nor reproduce the actions which brought a moderator to warn you.\\
When a user has already received multiple warning messages, other harsher sanctions may be applied.

\subsubsection{Role removal}
This sanction applies to users who own a non-self-assignable role. It consists in removing the special role from this user.\\
This sanction will be applied if one moderator is found guilty of moderation abuse.

\subsubsection{Blacklisting}
Blacklisting consists in forbidding any participation of a user to one or multiple event(s) or animation(s) organized by staff members for a certain duration (lifetime included).\\
This sanction may be applied if a user is found guilty of cheat in such an animation or event.

\subsubsection{Temporary mute}
Temporary mute is applied when you are being toxic in one or multiple discussions. It consists in removing your permission to talk in a channel for a pre-determined duration or until you have calmed down (in case of a rather violent debate with other users).

\subsubsection{Permanent mute}
Permanent mute is applied when you have already been warned or muted multiple times. This sanction can be cancelled by an administrator when you have proved you can behave politely and interact with others peacefully.\\
Getting permanently muted in more than 5 channels will result in a permanent ban from the server.\\\\
\textsl{{\small \textbf{Note:} The entire \texttt{VOICE CHANNELS} category counts as one unique entity: if you get muted in one of the voice channels, you get muted in all of them.}}

\subsubsection{Kicks}
Kicks consist in removing one user from the server whithout revoking his/her permission to join it again.\\
They are not very useful as the user can join immediately after being kicked, but their psychological impact is unexpectedly high as this is a strong proof that you can be removed from the community at any time.

\subsubsection{Temporary bans}
Temporary bans consist in removing one user from the server as well as revoking his/her permission to join it again for a certain amount of time. Their point is the same than for the kicks. They also allow the removal of the last messages that were posted by the user.

\subsubsection{Permanent bans}
They are the harshest sanctions of this list and they should only be used in cases of extreme emergency.\\
They consist in removing one user from the server as well as revoking his/her permission to join it again for an undetermined amount of time (it can be lifetime).\\
A permanent ban can be revoked by an admistrator if sufficient proofs have been brought to show the user was not guilty.\\
\\
This list is not exhaustive and some other sanctions may be decided by the moderators.\\
The choice of the appropriate sanction is let to the moderators' discretion.


\subsection{Reports}
If a user witnesses an abuse or violation of any of the rules set out in this document, he/she should report it via aprivate message, to a moderator or administrator as soon as possible.\\
\\
This is however not a duty and users who do not report abuses to staff members are not exposed to further prosecution.

\subsection{Complaints}
When a user faces harassment, defamation, threats, insults, intimidation, public release of his/her personnal information or any other unwanted and potentially dangerous situation for his reputation or physical or mental integrity, on Electronicity, he/she can send a complaint to a moderator or administrator via either a private message or a mentionon the server.\\
\\
Complaints will only be considered to be valid when they refer to an action that happened on Electronicity. All personnal drama between users should be kept out of the server and solved via private messages.\\
\\
Abusing of complaints  and monopolizing moderators and administrators’ attention with unnecessary drama may result in prosecution and sanctions.

\subsection{Appeals}
When a user is unhappy with the final verdict which was applied or when extra evidence of the innocence of sanctioned users is found, an appeal procedure can be started. There is no pre-determined procedure for such appeals as they will be treated individually.\\
\\
Appeal procedures can be refused by moderators or administrators if a sufficiently valid justification is brought. If an appeal is refused, the case is closed and no further investigation will be performed: the final decision is considered to be permanent.\\
\\
Multiple appeals can be requested for the same case as long as it is not closed by moderators or administrators.

\end{document}
